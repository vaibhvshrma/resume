% !TEX TS-program = xelatex
% !TEX encoding = UTF-8 Unicode
% -*- coding: UTF-8; -*-
% vim: set fenc=utf-8

%%%%%%%%%%%%%%%%%%%%%%%%%%%%%%%%%%%%%%%%%%%%%%%%%%%%%%%%%%%%%%%%%
%% CV.tex
%% <https://github.com/zachscrivena/simple-resume-cv>
%% This is free and unencumbered software released into the
%% public domain; see <http://unlicense.org> for details.
%%%%%%%%%%%%%%%%%%%%%%%%%%%%%%%%%%%%%%%%%%%%%%%%%%%%%%%%%%%%%%%%%

% See "README.md" for instructions on compiling this document.

\documentclass[letterpaper,MMMyyyy,nonstopmode]{simpleresumecv}
% Class options:
% a4paper, letterpaper, nonstopmode, draftmode
% MMMyyyy, ddMMMyyyy, MMMMyyyy, ddMMMMyyyy, yyyyMMdd, yyyyMM, yyyy

%%%%%%%%%%%%%%%%%%%%%%%%%%%%%%%%%%%%%%%%%%%%%%%%%%%%%%%%%%%%%%%%%
%% PREAMBLE.
%%%%%%%%%%%%%%%%%%%%%%%%%%%%%%%%%%%%%%%%%%%%%%%%%%%%%%%%%%%%%%%%%

% CV Info (to be customized).
\newcommand{\CVAuthor}{Vaibhav Sharma}
\newcommand{\CVTitle}{Vaibhav Sharma's CV}
\newcommand{\CVNote}{CV compiled on {\today}}
% \newcommand{\CVWebpage}{}

% PDF settings and properties.
\hypersetup{
pdftitle={\CVTitle},
pdfauthor={\CVAuthor},
% pdfsubject={\CVWebpage},
pdfcreator={XeLaTeX},
pdfproducer={},
pdfkeywords={},
unicode=true,
bookmarks=true,
bookmarksopen=true,
pdfstartview=FitH,
pdfpagelayout=OneColumn,
pdfpagemode=UseOutlines,
hidelinks,
breaklinks,
}


% Shorthand.
\newcommand{\Code}[1]{\mbox{\textbf{#1}}}
\newcommand{\CodeCommand}[1]{\mbox{\textbf{\textbackslash{#1}}}}

%%%%%%%%%%%%%%%%%%%%%%%%%%%%%%%%%%%%%%%%%%%%%%%%%%%%%%%%%%%%%%%%%
%% ACTUAL DOCUMENT.
%%%%%%%%%%%%%%%%%%%%%%%%%%%%%%%%%%%%%%%%%%%%%%%%%%%%%%%%%%%%%%%%%

\begin{document}

%%%%%%%%%%%%%%%
% TITLE BLOCK %
%%%%%%%%%%%%%%%

\Title{\CVAuthor}

\begin{SubTitle}
\par
\href{mailto:vaibhav180101@gmail.com}
{vaibhav180101@gmail.com}
\,\SubBulletSymbol\,
+91\,8233332555
\,\SubBulletSymbol\,
\href{https://linkedin.com/in/vaibhv-shrma}{https://linkedin.com/in/vaibhv-shrma}
\BigGap
Anyone can write code that a computer understands, I write such that \textbf{humans} do.
\Gap
I like code changes with <100 lines of code having meaningful tests. Love solving problems and engaging in productive discussions about tech. Strong
communicator and love working in a transparent environment.
\end{SubTitle}

\thispagestyle{empty}  % removes page number from this page

\begin{Body}

%%%%%%%%%%%%%%%
%% EXPERIENCE %%
%%%%%%%%%%%%%%%

\Section
{Experience}
{Experience}
{PDF:Experience}

\Entry
Software Engineer at\textbf{ Google, Bangalore}
\hfill 
\DatestampYMD{2022}{07}{04} --
Present
\BulletItem
Working in Payments Platform as a Java Backend Engineer on the eWallets team
\BulletItem
Team handles payments from eWallets across the world and handles millions of transactions monthly
\BulletItem
Goal to make payments easy for users for purchasing products and services from Google
\BulletItem
Worked on defining and building payment standards for vendors to allow users to top-up their eWallet balance while in a Google payment flow. This reduced transaction failures due to insufficient funds by upto 80\%.
\BulletItem
Worked on enabling users to pay with their Google linked eWallets on 3P merchant websites (e.g. nike.com) pushing forward GPay Wallet as the all-in-one payment solution for users in APAC markets.
\BulletItem
Migrated Paypal add FOP journey of the users across the world to the latest generation of server driver UI while increasing add FOP success rate by 4\% resulting in incremental revenue generation for Google.
\BulletItem
\textbf{Extra work:} At Google, experiments are used to control and incrementally launch new features and products. Setting up experiments across multiple product layers to work in sync requires significant dev-effort to get done correctly. Built a tool to automate setting up such complex experiments. This helped increase developer velocity and reduced dev-effort down from around a week to a day. 

\BigGap
\Entry
Software Engineer at\textbf{ HackerEarth, Bangalore}
\hfill 
\DatestampYMD{2020}{08}{01} --
\DatestampYMD{2022}{07}{31}
\BulletItem
Implemented Points and Badges to game-ify HackerEarth Community. Broke down complex requirements into easy to maintain and extend code with efficient APIs. \href{https://www.hackerearth.com/docs/wiki/developers/points-and-badges-on-hackerearth/}{Read about it by clicking here}.
\BulletItem
Built the full stack for \href{https://www.hackerearth.com/practice/problems/}{list view for practice questions (click here)}. Highly performant APIs coupled with a beautiful interface created using the following. 
\SubBulletItem Frontend: React  
\SubBulletItem Backend: Django Rest Framework-Python
\BulletItem
Built DIY-Listed Tests which allows recruiters to list tests on HackerEarth
community for higher test take-up rate. \href{https://help.hackerearth.com/hc/en-us/articles/4412341460633-Listed- tests}{Read about it by clicking here}. 

%%%%%%%%%%%%
%% SKILLS %%
%%%%%%%%%%%%

\Section
{Skills}
{Skills}
{PDF:Skills}

\Entry
Java,
Python,
Django,
Django Rest framework,
React,
Data Structures and Algorithms,
MySQL,
Object Oriented Programming,
Testing

\Section
{Education}
{Education}
{PDF:Education}

\Entry
\textbf{B.Tech Computer Science}, Manipal University Jaipur
\hfill  2016 -- 2020
\BulletItem
\textbf{CGPA 9.63}

\BigGap
\Entry
\textbf{Delhi Public School, Jaipur}
\BulletItem
CBSE All India Senior School Certificate Exams - Class 12
\hfill 2015
\SubBulletItem
89.8\% in English + PCM + Information Practices
\Gap
\BulletItem
CBSE All India Secondary School Examination - Class 10
\hfill 2013
\SubBulletItem
CGPA: 9.8

%%%%%%%%%%%%%%%%%%%%%%%%%%%
%% AWARDS & SCHOLARSHIPS %%
%%%%%%%%%%%%%%%%%%%%%%%%%%%

\Section
{Awards \&\newline
Recognitions}
{Awards \& Recognitions}
{PDF:AwardsAndRecognitions}

\BulletItem
Multiple Peer Bonuses at Google
\begin{Detail}
\Item
For helping various team mates in their work in the form of ramping up, sharing expertise, fixing critical bugs, unblocking launches, etc
\Item
Peer bonus by former manager for building the experiment automation tool, which helped launch in multiple regions quickly. 
\end{Detail}

\BulletItem
Winner of HackerEarth's The Great Hackathon \#19
\begin{Detail}
\Item
Won 1st place in HackerEarth's internal hackathon. Implemented Skill
Badges for the users. Lead of team of 5.
\end{Detail}

\BulletItem
HackerEarth Mission Impossible Award for high quality delivery Q2-2021
\begin{Detail}
\Item
For driving gamification end-to-end and ensuring 0 bugs in QA and Prod.
\end{Detail}

\BulletItem
HackerEarth Kudos Award for best performing new comer Q3-2020

\end{Body}

\BigGap
\UseNoteFont%
\null\hfill%
[\textit{\CVNote}]



\end{document}